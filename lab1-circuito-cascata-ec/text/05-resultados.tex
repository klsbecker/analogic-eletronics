\section{Resultados}

A seguir, apresentamos os resultados obtidos a partir das medições realizadas no circuito amplificador em cascata. Os dados foram organizados em tabelas para facilitar a comparação entre os valores teóricos, simulados e medidos.

\subsection{Ponto Quiescente}
\begin{center}
    \begin{tabular}{|c|c|c|c|c|}
        \hline
        \textbf{Transistor} & \textbf{$I_C$ Calculado} & \textbf{$V_{CE}$ Calculado} & \textbf{$I_C$ Medido} & \textbf{$V_{CE}$ Medido} \\
        \hline
        Q1 & 1,5~mA & 4,7~V & 1,4~mA & 4,8~V \\
        Q2 & 1,5~mA & 4,7~V & 1,5~mA & 4,9~V \\
        \hline
    \end{tabular}
\end{center}

\subsection{Ganho de Tensão COM CARGA em Cada Estágio - Calculado}
\begin{center}
    \begin{tabular}{|c|c|c|c|c|c|}
        \hline
        $v_{B1}$ & $v_{C1}$ & $A_{v1}$ & $v_{B2}$ & $v_{C2}$ & $A_{v2}$ \\
        \hline
        6,8~mV & 1,2~V & 176,5 & 1,2~V & 3,1~V & 2,5 \\
        \hline
    \end{tabular}
\end{center}

\subsection{Ganho de Tensão COM CARGA em Cada Estágio - Simulado}
\begin{center}
    \begin{tabular}{|c|c|c|c|c|c|}
        \hline
        $v_{B1}$ & $v_{C1}$ & $A_{v1}$ & $v_{B2}$ & $v_{C2}$ & $A_{v2}$ \\
        \hline
        6,9~mV & 1,3~V & 188,4 & 1,3~V & 2,9~V & 2,2 \\
        \hline
    \end{tabular}
\end{center}

\subsection{Ganho de Tensão COM CARGA em Cada Estágio - Medido}
\begin{center}
    \begin{tabular}{|c|c|c|c|c|c|}
        \hline
        $v_{B1}$ & $v_{C1}$ & $A_{v1}$ & $v_{B2}$ & $v_{C2}$ & $A_{v2}$ \\
        \hline
        19,7~mV & 2,4~V & 121,8 & 2,4~V & 6,2~V & 2,6 \\
        \hline
    \end{tabular}
\end{center}

\subsection{Ganho de Tensão SEM CARGA em Cada Estágio - Calculado}
\begin{center}
    \begin{tabular}{|c|c|c|c|c|c|}
        \hline
        $v_{B1}$ & $v_{C1}$ & $A_{v1}$ & $v_{B2}$ & $v_{C2}$ & $A_{v2}$ \\
        \hline
        6,8~mV & 1,5~V & 220,6 & 1,2~V & 4,6~V & 3,8 \\
        \hline
    \end{tabular}
\end{center}

\subsection{Ganho de Tensão SEM CARGA em Cada Estágio - Simulado}
\begin{center}
    \begin{tabular}{|c|c|c|c|c|c|}
        \hline
        $v_{B1}$ & $v_{C1}$ & $A_{v1}$ & $v_{B2}$ & $v_{C2}$ & $A_{v2}$ \\
        \hline
        6,8~mV & 1,7~V & 250 & 1,2~V & 4,2~V & 3,5 \\
        \hline
    \end{tabular}
\end{center}

\subsection{Ganho de Tensão SEM CARGA em Cada Estágio - Medido}
\begin{center}
    \begin{tabular}{|c|c|c|c|c|c|}
        \hline
        $v_{B1}$ & $v_{C1}$ & $A_{v1}$ & $v_{B2}$ & $v_{C2}$ & $A_{v2}$ \\
        \hline
        19.6~mV & 2,4~V & 123,5 & 2,4~V & 9,0~V & 3,8 \\
        \hline
    \end{tabular}
\end{center}

\subsection{Ganho de Tensão COM CARGA e SEM CARGA - Comparativo}

\begin{center}
    \begin{tabular}{|c|c|c|c|c|c|c|}
        \cline{2-7}
        \multicolumn{1}{c|}{} & \multicolumn{3}{c|}{\textbf{COM CARGA}} & \multicolumn{3}{c|}{\textbf{SEM CARGA}} \\
        \hline
        \textbf{Ganho} & \textbf{Calculado} & \textbf{Simulado} & \textbf{Medido} & \textbf{Calculado} & \textbf{Simulado} & \textbf{Medido} \\
        \hline
        \textbf{$A_{v1}$} & 176,5 & 188,4 & 121,8 & 220,6 & 250 & 123,5 \\
        \hline
        \textbf{$A_{v2}$} & 2,5 & 2,2 & 2,6 & 3,8 & 3,5 & 3,8 \\
        \hline
    \end{tabular}
\end{center}
