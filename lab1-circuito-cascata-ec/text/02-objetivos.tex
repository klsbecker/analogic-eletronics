\section{Referêncial Teórico}

\subsection{Amplificação de Sinais}
A amplificação é um processo que aumenta a amplitude de um sinal elétrico. O ganho é expresso em decibéis (dB) e depende da configuração dos componentes ativos e da topologia do circuito.

\subsection{Transistores Bipolares}
Os transistores bipolares de junção (BJTs) são dispositivos semicondutores amplamente utilizados em circuitos de amplificação devido à sua capacidade de fornecer ganho de corrente e tensão. Eles operam em três regiões: corte, ativa e saturação. Para aplicações de amplificação, o transistor deve operar na região ativa, onde a corrente de coletor é proporcional à corrente de base, conforme o ganho de corrente ($\beta$) do dispositivo.

\subsection{Ponto de Operação}

O ponto de operação (Q-point) define o regime de funcionamento do transistor no amplificador. Ele é determinado pela corrente de coletor ($I_C$) e pela tensão coletor-emissor ($V_{CE}$) no estado de repouso. Um Q-point bem posicionado garante que o transistor opere de forma linear, evitando distorções no sinal amplificado.

\subsection{Configuração Emissor-Comum}
Na configuração emissor-comum, a tensão de entrada é aplicada à base do transistor e a saída é obtida no coletor. Essa configuração fornece alto ganho de tensão e inversão de fase entre entrada e saída.

\subsection{Amplificadores em Cascata}
Múltiplos estágios amplificadores em cascata multiplicam o ganho total. No entanto, efeitos como capacitâncias parasitas podem influenciar a resposta em frequência, limitando a faixa operacional do amplificador.
