\section{Introdução}

Amplificadores são circuitos essenciais para condição e processamento de sinais em sistemas eletrônicos. A topologia de amplificadores em cascata permite aumentar significativamente o ganho sem comprometer a integridade do sinal, sendo amplamente utilizada em aplicações como comunicação e instrumentação.

Este experimento tem como objetivo analisar o comportamento de um circuito amplificador composto por dois estágios em cascata utilizando transistores na configuração emissor comum. Serão avaliados aspectos como ganho total, resposta em frequência e impacto da polarização na estabilidade do circuito.

O estudo compreende a simulação computacional, a montagem do circuito e a comparação entre os resultados práticos e teóricos, buscando validar o funcionamento e compreender suas limitações.