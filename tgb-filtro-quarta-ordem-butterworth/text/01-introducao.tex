\section{Introdução}

A manipulação de sinais analógicos exige condicionamento preciso, onde filtros são cruciais para a seletividade de frequência. Este trabalho aborda o projeto de um filtro passa-baixas VCVS (Voltage-Controlled Voltage Source) de quarta ordem, empregando a aproximação de Butterworth para garantir uma resposta de fase linear e atenuação maximamente plana na banda passante. A ordem do filtro assegura uma transição acentuada, com frequência de corte definida em 3 kHz e ganho dimensionado conforme a especificação do sinal.