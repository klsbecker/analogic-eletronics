\section{Introdução}

Amplificadores operacionais são blocos fundamentais em sistemas analógicos, permitindo a implementação de funções matemáticas como integração e diferenciação de sinais. As configurações de integrador e diferenciador são amplamente aplicadas em instrumentação, controle e processamento de sinais.

Este experimento tem como objetivo analisar o comportamento de circuitos integrador e diferenciador utilizando amplificadores operacionais. Serão investigados aspectos como frequência de corte, resposta em frequência e forma de onda de saída para diferentes sinais de entrada.

O estudo envolve a montagem prática dos circuitos, análise das formas de onda e comparação entre os resultados teóricos e experimentais, visando compreender o funcionamento e as limitações reais dessas topologias.